\section{Related Works}


Suggested features recognized by cardiologists \cite{reiffel2010practice}
Published methods for analysis of absence of P waves or the presence of fibrillatory f waves in the TQ interval

Echo state network \cite{petrenas2012echo}
P-wave absence (PWA) based detection \cite{ladavich2015rate}
analysis of the average number of f waves \cite{du2014novel}
P-wave-based insertable cardiac monitor application \cite{purerfellner2014p},
wavelet entropy \cite{alcaraz2006wavelet} \cite{rodenas2015wavelet} and wavelet energy \cite{garcia2016application}.

TAKEN FROM https://physionet.org/challenge/2017/:
"Previous studies concerning AF classification are generally limited in applicability because 1) only classification of normal and AF rhythms were performed, 2) good performance was shown on carefully-selected often clean data, 3) a separate out of sample test dataset was not used, or 4) only a small number of patients were used. It is challenging to reliably detect AF from a single short lead of ECG, and the broad taxonomy of rhythms makes this particularly difficult. In particular, many non-AF rhythms exhibit irregular RR intervals that may be similar to AF. In this Challenge, we treat all non-AF abnormal rhythms as a single class and require the Challenge entrant to classify the rhythms as 1) Normal sinus rhythm, 2) AF, 3) Other rhythm, or 4) Too noisy to classify."
