%\documentclass[fontsize=11pt, appendixprefix=true]{scrreprt}
\documentclass[11pt]{report}
\usepackage{tocloft}
%\renewcommand{\cftpartleader}{\cftdotfill{\cftdotsep}} % for parts
%\renewcommand{\cftchapleader}{\cftdotfill{\cftdotsep}} % for content
\renewcommand{\cftsecleader}{\cftdotfill{\cftdotsep}} % for sections, if you really want! (It is default in report and book class (So you may not need it).

% appendixprefix: hogy odaírja, hogy "Függelék A", ne csak "A"
%\usepackage[english, magyar]{babel}                        % nyelvi csomag
\usepackage[T1]{fontenc}                                   % ékezetes betűknél is legyen automatikus elválasztás
\usepackage[utf8]{inputenc}                                % ékezetes betűk kezelése
%\usepackage{lmodern}                                       % alapértelmezett betűtípus ne legyen pixeles
%\usepackage{mathtools}                                     % képletekhez kell
\usepackage[backend=biber, sorting=anyt]{biblatex}           % bibliográfia
\usepackage{parskip}
\addbibresource{fot.bib}
\addbibresource{zotero.bib}
\usepackage{url}
\usepackage{enumitem}
\usepackage{amsmath}
\usepackage{amssymb}
\usepackage{graphicx}                                      % képek beszúrása
%\usepackage{subcaption}
\usepackage{subfig}
\graphicspath{ {images/} }
\usepackage[export]{adjustbox}                             % ez az ITK logó pozicionálásához kell
\usepackage[margin=2.5cm, bindingoffset=1.25cm]{geometry}  % margók
\usepackage[onehalfspacing]{setspace}                      % másfeles sorköz
\usepackage[hidelinks, unicode, pdfusetitle]{hyperref}     % kattintható tartalomjegyzék és hivatkozások
\usepackage{bookmark}                                      % PDF könyvjelzők
\usepackage{csquotes}                                      % a bibliográfiában megfelelően legyenek formázva az idézőjelek
\DeclareQuoteAlias{german}{magyar}
\usepackage{blindtext} %mer mért ne?
% Kódrészletekhez ajánlom
\usepackage{listings, scrhack}
%\usepackage{sourcecodepro} % egy jó betűtípus
\lstset{captionpos=b, numberbychapter=false, basicstyle=\ttfamily, showstringspaces=false, columns=fullflexible}
% Kódrészletek magyar stílusú számozása
%\renewcommand\lstlistingname{kódrészlet}
%\makeatletter
%\renewcommand\fnum@lstlisting{\ifx\lst@@caption\@empty\else\thelstlisting.~\fi\lstlistingname}%
%\makeatother
% Nyilatkozathoz két parancs definíciója
\newcommand{\pushtobottom}{\vspace*{\fill}}
\newcommand{\signatureline}[1]{\begin{flushright}
	\vspace*{.5cm}\par\noindent\makebox[2.5in]{\hrulefill}
	\par\noindent\makebox[2.5in][c]{#1}
	\end{flushright}
}

% reference
%\usepackage[plain]{fancyref}
%\usepackage[english, nameinlink]{cleveref}

%formázás

\usepackage{fancyhdr}
\pagestyle{fancy}
\fancyhf{}
%\fancyhead[L]{\leftmark}
\fancyhead[L]{\textsl{\rightmark}}
\fancyhead[R]{\thepage}
\renewcommand{\headrulewidth}{.001cm}

% Ezeket írd át!
\author{Csaba Botos}
\title{
Deep learning framework in TensorFlow\\
for biomedical research
}
\date{2017}

\usepackage{titlesec}

\usepackage{courier}
\lstset{basicstyle=\ttfamily,breaklines=true}
\lstset{framextopmargin=50pt}

% Chapter customization
\renewcommand\thesection{\arabic{section}}
\setcounter{secnumdepth}{2}

\titleformat{\chapter}[block]
  {\normalfont\Huge\bfseries}{\thechapter.}{1em}{\huge}
%\titleformat{\subsection}[block]

% For quotes
\usepackage{epigraph}
\setlength\epigraphwidth{0.7\textwidth}
\setlength\epigraphrule{0pt}

% For multi fig in row
\usepackage{floatrow}

% No page break before chapters
\usepackage{etoolbox}
\makeatletter
%\patchcmd{\chapter}{\if@openright\cleardoublepage\else\clearpage\fi}{}{}{}
\makeatother

\begin{document}
%%%%%%%%%%%%%%%%%%%%%%%%%%%%%%%
%% BEVEZETÉS

\includegraphics[valign=m]{ITK_logo} \parbox[c]{\textwidth}{
Pázmány Péter Catholic University\\
Faculty of Information Technology and Bionics}
\vspace*{\fill}

{\let\newpage\relax\maketitle}
\vspace*{\fill}
\begin{center}
A review submitted in partial satisfaction of the requirements of Guided Individual Study.
\bigskip

Advisor: PhD.\ István Z. Reguly \\
\end{center}
\clearpage

\pagenumbering{Roman}
\chapter*{Abstract}
Atrial Fibrillation is the most common type of cardiac arythmia.

ötletek még abstracthoz

http://www.sciencedirect.com/science/article/pii/S1746809413000062

https://physionet.org/challenge/2017/

\addcontentsline{toc}{chapter}{Abstract}
\tableofcontents
\addcontentsline{toc}{chapter}{Bibliography}
\clearpage

\pagenumbering{arabic}
\section{Introduction}


\paragraph{Social relevance.}
Atrial Fibrillation is the most common type of cardiac arrhythmia.
Thousands of heart failures could be prevented by proper treatment if early signs were diagnosed in time.
Over the years medical equipment advanced by involving some sort of artificial intelligence.
We don't have to go far away, probably every gas station in the area has a semi-automatic defibrillator, that has a sensor built in which recognizes whether the patient needs to be shocked or not --- to prevent unnecessary reanimation.
The basic idea is to relieve overwhelmed doctors by recognizing invariant patterns in the specific cases.
These patterns are taught in medical universities, and fine-tuned during years of practice --- but turns out that volunteering cardiologists can submit their knowledge bases to engineers who in return will automatize at least the trivial process to support the better treatment.
In order to assist in the diagnosis, and make predictions based on previous cases we utilize Machine Learning techniques to combine professional knowledge and neural networks to achieve the lowest error rate on a classifying task.

\paragraph{Biological background.}
\textbf{TODO}

\paragraph{The real challenge.}
This year the PhysioNet Challenge~\cite{physionet} is simple, a set of single lead ECG samples labeled by a team of cardiologists is provided. The labels are the following: \textit{noisy, normal, atrium fibrillation, other}. The task is to develop a method that is able to classify from an unseen prerecorded sample.
At first sight the initial conditions are very encouraging, but soon after the first check of the training files it turns out that only a limited number of samples are given, exactly 8528 samples, that has a strongly unbalanced distribution between classes.~\ref{original-class-hist}
\ref{updated-class-hist}.
Also as it was revealed during the last weeks, the ground truth was error-prone, misleading the training process.
We had several discussions in the early phase of development with professionals about possible classification errors, that even were trivial to us~\ref{trivial error}. Many suggestions were committed by the competitors, and as a result the organizers released an updated label reference.


%% BEVEZETÉS
%%%%%%%%%%%%%%%%%%%%%%%%%%%%%%%
%% TARTALOM
%\chapter{Related Works}


Suggested features recognized by cardiologists~\cite{reiffel2010practice}
Published methods for analysis of absence of P waves or the presence of fibrillatory f waves in the TQ interval

Echo state network~\cite{petrenas2012echo}
P-wave absence (PWA) based detection~\cite{ladavich2015rate}
analysis of the average number of f waves~\cite{du2014novel}
P-wave-based insertable cardiac monitor application~\cite{purerfellner2014p},
wavelet entropy~\cite{alcaraz2006wavelet, rodenas2015wavelet} and wavelet energy~\cite{garcia2016application}.

TAKEN FROM https://physionet.org/challenge/2017/:
"Previous studies concerning AF classification are generally limited in applicability because 1) only classification of normal and AF rhythms were performed, 2) good performance was shown on carefully-selected often clean data, 3) a separate out of sample test dataset was not used, or 4) only a small number of patients were used. It is challenging to reliably detect AF from a single short lead of ECG, and the broad taxonomy of rhythms makes this particularly difficult. In particular, many non-AF rhythms exhibit irregular RR intervals that may be similar to AF. In this Challenge, we treat all non-AF abnormal rhythms as a single class and require the Challenge entrant to classify the rhythms as 1) Normal sinus rhythm, 2) AF, 3) Other rhythm, or 4) Too noisy to classify."

Pan Tompkins QRS detection~\cite{waser2013removing}.

Combining Machine Learning techniques~\cite{geurts2001pattern}

Deep Learning for time series classification~\cite{langkvist2014review, wang2016time}


\section{My commitments to the project}

I aim to generalize this writing and be less specific about the exact project we are working on, so the following experiences can be transferred to other fields of otherwise non related studies.
Basically our task was a simple classification problem for time samples of various length.
My task was not just to try out famous Machine Learning concepts which \it{may} work, but to support our research group with the necessary backend of the experiments carried out, and to integrate different results in the final unified model.

\subsection{Environment}

For doing so I vouched for using TensorFlow environment, in which I had previous experience \cite{github-projects}.
For our project we started by applying different previously proven to be succesful feature extraction methods, and for introducing Deep Learning we wanted to improve our progress by building on top of these features.
So in the first step I had to break down the recent DL architectures I wanted to use to basic modules.
When reimplementing them I paid attention to make spaceholders or entry points for every possible external features my colleagues developed.

Entry points at different level of processing the input can be categorized as the following:
- Variable length features (i.e. output of filters).
- Fixed representation features (i.e. variability indices, frequency domain components).
- Suggestions for classes (i.e. external classifier suggestion)

Also, to improve the model's efficiency I applied queues after different entry points to make mini-batch processing available, by filling these queues independently in mutliple threads, so the network trainer do not have to wait for these external sources to finish their processes to update the network's weights.

\subsection{Data standardization}

The biggest obstacles at first sight in using TensorFlow are the Tensors and the Flows themselves.


%% TARTALOM
%%%%%%%%%%%%%%%%%%%%%%%%%%%%%%%
%% BEFEJEZÉS
\chapter{Results}
Here we list the evaluation scores, and further guides we use for planning our next steps, available at the time of writing.

\section{Fully Convolutional Network}
Our overall impression with convolutional networks is that, they are slowly training, even high capacity networks are unable to achieve satisfactory score on training set~\ref{fig:convnet-overall}. For this reason, we aimed first to use an architecture, that is capable to over-fit on the train data, so we could fine-tune the training by enforcing the regularization settings.
\paragraph{MLP problem.}
A strange result is that by adding MLP block before the classifier, the performance was drastically reduced, and soon the training collapsed.
The models were biased towards only choosing a single class.
Excellent example for this behavior can be seen on Figure \ref{fig:convnet-bad}, where models with MLP built in oscillate around the same performance throughout an \textit{epoch}. (epoch: a set of training steps which includes every entry from the training set.)
We can also check out the confusion matrix of these models
on Figure \ref{fig:bad-conf-op}, also the histogram of the model parameters reveals, that the network is strongly biased towards choosing one specific class independently from its input in Figure \ref{fig:bad-hist}.

\begin{figure}[h]
  \centering
  \includegraphics[width=\textwidth]{convnet-overall}
  \caption{Overall performance of the FCN approaches plotted in TensorBoard. Top: Accuracy derived from the confusion operator. Bottom: Unweighted loss during training}
  \label{fig:convnet-overall}
\end{figure}

\section{Residual Network}
Our next choice were deep residual networks.
With this model, we could achieve better training and evaluation scores.
The exclusion of the MLP block had to be made when using ResNets also, because the same behavior occurred when we applied hidden layers between feature extractors and the classifier layer.

\begin{figure}[h]
  \centering
  \includegraphics[width=\textwidth]{resnet-overall}
  \caption{Overall performance of the ResNet approaches plotted in TensorBoard. Top: Accuracy on the training set. Middle: Accuracy on the evaluation set. Bottom: Unweighted loss during training}
  \label{fig:resnet-overall}
\end{figure}


\section{Future}

\paragraph{Short-term plans.}
Continue training with different parameters
Try to evaluate samples by plotting them first

\paragraph{Long-term plans}
Utilize OneShot learning to tackle the small-train set problem.
Implement DANN training to transfer our best model's trained parameters to real world applications which may introduce different sample density.

\clearpage
\printbibliography
%% BEFEJEZÉS
%%%%%%%%%%%%%%%%%%%%%%%%%%%%%%%
\end{document}
