We have developed an algorithm combining machine learning and handcrafted
features with medical relevances.
Our architecture consist of three main parts:

\paragraph{Deep Learning.} using a
convolutional neural network for initial feature extraction, reserving the
temporal aspects of the data. These time dependent features are fed into
the recurrent network which handles the temporal aspects of the data and
generates a feature vector with fixed length.

\paragraph{QRS complex filtering.}
Another set of features is based on the Pan Tompkins algorithm to identify the
characteristic QRS descriptors of the data. The variance of the detected QRS
positions along the variability index of P and T waves, which describe the morphology of the whole cardiac cycle, are also used in the classification.

\paragraph{Frequency domain analysis.} The last set of handcrafted features uses Fourier descriptors of the data.
The Fourier domain of the training data was analyzed and those frequencies which
show a large difference between normal and fibrillation cases were selected. The
biological relevance of the selected frequencies was examined and advised by
cardiologists.

In the current implementation we concatenate the feature vectors coming from the
three previously described parts and these features are classified by a Multi Layer Perceptron.

\textit{\textbf{Keywords: deep learning, residual network, fully convolutional network, time series, recurrent network, transfer learning}}
