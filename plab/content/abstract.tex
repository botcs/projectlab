Workgroup's abstract:

"We have developed an algorithm combining machine learning a handcrafted
features with medical relevances.

Our method combines the popular approaches of convolutional (CNN) and recurrent
neural networks (RNN), these state of the art methods are providing a set of
deep learning features, which are extended by two sets of handcrafted features
coming from traditionaly applied method like the variability indices of the Pan
Tompkins algorithm and Fourier descriptors.

Our architecture consist of three main parts: 1, Machine learning part using a
convolutional neural network for initial feature extraction, reserving the
temporal aspects of the data. These initial, temporal features are fed into a
Long short-term memory which handles the temporal aspects of the data and
generates a feature vector with fixed length.

2, One set of features is based on the Pan Tompkins algorithm to identify the
characteristic QRS descriptors of the data. The variance of the detected QRS
positions along with other indices (e.g. variabilitiy of P and T wave
detections) which describe the morphology of the whole cardiac cycle better are
also used in the classification.

3, The last set of handcrafted features uses Fourier descriptors of the data.
The Fourier domain of the training data was analysed and those frequencies which
show a large difference between normal and fibrillation cases were selected. The
biological relevance of the selected frequencies was examined and advised by
cardiologists.

In the current implementation we concatenate the feature vectors coming from the
three previously described parts and these features are classified by a fully
connected neural network. The initial netwrok with random weights was tested and
the optimization of the paramteres of the algorithm (e.g.: complexity of the CNN
and RNN part) is ongoing. The algorithm was implemented in Tensorflow and
exploits the multi-parallel architecture of GP-GPUs."
